A metodologia adotada nesta pesquisa se fundamenta na análise do conjunto de dados \textit{"Loan Data for Dummy"}, 
visando a compreensão e modelagem de padrões associados a operações de empréstimos. Dois métodos distintos, 
Regressão Logística e Redes Neurais, serão empregados para investigar as relações existentes nos dados e aprimorar as previsões.
A implementação desses modelos será realizada utilizando tanto a linguagem de programação R quanto Python,
permitindo uma abordagem comparativa e uma compreensão abrangente das nuances de cada implementação.
Além disso, a técnica SHAP (\textit{Shapley Additive Explanations}) será integrada para proporcionar
uma interpretação aprofundada dos modelos, ampliando a transparência nas decisões preditivas. 
Essa abordagem multidimensional busca fornecer insights valiosos sobre o comportamento do conjunto de dados, 
enriquecendo a compreensão dos resultados obtidos com diferentes técnicas de modelagem.


\subsection{Conjunto de dados}

O banco de dados "Loan Data for Dummy" é uma base de dados do Kaggle, projetada para simular informações relacionadas 
a operações de empréstimos. Desenvolvido para fins educacionais e de pesquisa, esse conjunto tem sua origem de um modelo de 
banco \textit{"peer to peer"} sediado na Irlanda, no qual o banco disponibiliza recursos a potenciais clientes, 
obtendo lucros com base no risco que assume. 
Os dados disponíveis no Kaggle representam uma versão fictícia de uma situação real, com a maior parte dos dados manipulados
ou criados sinteticamente para preservar as informações dos clientes originais.

A variável que será foco do estudo é a "loan condition". Essa variável oferece insights cruciais,
uma vez que reflete a condição do empréstimo, desempenhando um papel central na análise. Através dessa variável, é 
possível discernir se um empréstimo foi classificado como "bom" ou "mau", proporcionando uma avaliação da qualidade e 
risco associados a cada transação. No contexto deste conjunto de dados, presume-se que a "loan condition" seja uma variável binária,
onde, por exemplo, "0" poderia indicar um empréstimo em boas condições e "1" indicaria o contrário.
A compreensão aprofundada dessa variável é essencial para a construção e interpretação adequada dos modelos subsequentes, 
como a regressão logística e redes neurais, permitindo uma análise mais precisa e informada do risco associado aos empréstimos.

Para se avaliar a variável "loan condition" foi utilizada algumas variáveis presentes na base de dados, como:

\begin{enumerate}
  \item \textbf{Tempo de emprego}: Representa o tempo de emprego do solicitante expresso numericamente. Um valor de 5 indicaria que o indivíduo está empregado há 5 anos.
  \item \textbf{Tipo de residência}: Indica o status de moradia do solicitante, como proprietário, inquilino ou outra forma de ocupação residencial.
  \item \textbf{Renda anual}: Reflete a renda anual do solicitante, uma medida crucial para avaliar a capacidade de pagamento do empréstimo. Pode ser expressa numericamente, por exemplo, 50,000.
  \item \textbf{Valor do empréstimo}: Representa o valor do empréstimo solicitado pelo requerente, geralmente expresso em termos monetários, como 10,000.
  \item \textbf{Prazo}: Indica o prazo do empréstimo, especificando o período de tempo durante o qual o empréstimo deve ser reembolsado. Pode ser, por exemplo, 36 meses.
  \item \textbf{Tipo de aplicação}: Refere-se ao tipo de aplicação, indicando se é uma aplicação individual ou conjunta.
  \item \textbf{Finalidade}: Descreve a finalidade do empréstimo, como consolidação de dívidas, compra de casa, educação, entre outros.
  \item \textbf{Tipo do juros}: Indica a natureza dos pagamentos de juros, se são fixos ou variáveis.
  \item \textbf{Taxa de juros}: Representa a taxa de juros associada ao empréstimo, geralmente expressa como uma porcentagem, como 10.
  \item \textbf{Grau}: Refere-se à classificação de risco do tomador de empréstimo atribuída pela instituição financeira, como A, B, C, etc.
  \item \textbf{DTI}: Significa "Debt-to-Income" (Dívida-para-Renda) e representa a proporção entre as dívidas mensais e a renda mensal do requerente, proporcionando uma medida da capacidade de pagamento.
  \item \textbf{pagamento\_bruto}: Representa o valor total pago, incluindo o principal e os juros, ao final do empréstimo.
  \item \textbf{pagamento\_liquido}: Indica o total de principal (quantia inicial do empréstimo) recuperado até o momento.
  \item \textbf{Valor recuperado}: Representa o valor recuperado em caso de inadimplência ou perda.
  \item \textbf{Parcelas}: Refere-se à parcela mensal que o requerente do empréstimo deve pagar, incluindo tanto o principal quanto os juros.
  \item \textbf{Região}: Indica a região geográfica associada ao requerente do empréstimo.
\end{enumerate}


% TODO: fazer fluxograma para o modelo logístico
% TODO: fazer fluxograma para a rede neural


\subsection{Métodos}

Falar da regressão logística aplicada na base de dados,
como será feito a preparação do modelo, normalização etc

Falar do modelo de redes neurais utilizado, arquitetura base

Detalhar o porquê de se interpretar os modelos e as maneiras
que é feito isso(testes, coefs, shap)

O intuito é ser a ponte entre referencial teório e resultados
