\subsection{Análise descritiva}

\subsubsection{Condição do empréstimo}

A variável "Condição do empréstimo" é a variável reposta desse estudo, como foi definido anteriormente.
Com isso temos o seguinte comportamento dessa variável:

\begin{table}[H]
  \centering
  \begin{tabular}{lcc}
    \hline
    \textbf{Condição do empréstimo} & \textbf{Número de observações} & \textbf{Frequência relativa} \\ \hline
    Empréstimo bom  & 819950 & 92,4\% \\
    Empréstimo ruim & 67429  & 7,59\% \\ \hline
  \end{tabular}
  \caption{Número de observação em cada categoria da variável resposta}
  \label{tab:freq_var_res}
\end{table}

A Tabela \ref{tab:freq_var_res} mostra a distribuição da variável "Condição do empréstimo". Uma variável composta majoritariamente 
por observações do tipo "Empréstimo bom", onde a mesma está presente em mais de 90\% das observações na base de dados, mostrando
que a cada 12 empréstimos rotulados como "bons", existe 1 rotulado como "ruim". 

\subsubsection{Relação entre as covariáveis e a variável resposta}

% boxplot
\begin{figure}[H]
  \centering
  \subImagem{imagens/exploratoria/quantidade_emprestimo.jpg}{Valor do empréstimo}
  \vspace{.5cm}
  \subImagem{imagens/exploratoria/duracao_emprestimo.jpg}{Tempo de emprego (em anos)}
  \vspace{.5cm}
  \subImagem{imagens/exploratoria/parcela.jpg}{Valor da parcela do emprésitmo}
 \subImagem{imagens/exploratoria/renda_anual.jpg}{Renda anual (em dólar)}
  
  \label{fig:exp_boxplot1}
  \caption{Variáveis explicativas em relação à condição do empréstimo}
  
\end{figure}

O comportamento da variável resposta nas Figuras \ref{imagens/exploratoria/quantidade_emprestimo.jpg} 
e \ref{imagens/exploratoria/parcela.jpg} demonstrou semelhanças, onde, em ambos os casos,
não foi evidenciada uma clara diferença entre o valor do empréstimo e o valor da parcela em relação às
categorias da variável resposta. A Figura \ref{imagens/exploratoria/duracao_emprestimo.jpg} 
também apresenta um comportamento semelhante entre as classes
"Empréstimo ruim" e "Empréstimo bom", mas com um detalhe: a mediana do tempo de trabalho dos clientes rotulados 
como "Empréstimo ruim" foi inferior em comparação ao outro caso. Por fim, a Figura \ref{imagens/exploratoria/renda_anual.jpg} 
indica que clientes 
com uma renda anual elevada tendem a ser categorizados como "Empréstimo bom".


\begin{figure}[H]
  \centering
  \subImagem{imagens/exploratoria/taxa_juros.jpg}{Taxa de juros do empréstimo}
  \vspace{.5cm}
  \subImagem{imagens/exploratoria/dti.jpg}{Razão entre a dívida e o salário do cliente}
  \vspace{.5cm}
  \subImagem{imagens/exploratoria/pagamento_bruto.jpg}{Valor bruto do emprétimo pago}
 \subImagem{imagens/exploratoria/pagamento_liquido.jpg}{Valor líquido do emprétimo pago}
 \caption{Variáveis explicativas em relação à condição do empréstimo}
  
 \label{fig:exp_boxplot2}
\end{figure}

A Figura \ref{imagens/exploratoria/taxa_juros.jpg} evidencia uma relação significativa entre taxas de juros elevadas e empréstimos 
considerados ruins. A Figura \ref{imagens/exploratoria/dti.jpg} complementa a informação fornecida pela 
Figura \ref{imagens/exploratoria/renda_anual.jpg},
indicando que clientes com renda mais elevada tendem a cumprir adequadamente com seus pagamentos.
As Figuras \ref{imagens/exploratoria/pagamento_bruto.jpg} e  \ref{imagens/exploratoria/pagamento_liquido.jpg} 
seguem padrões semelhantes, sugerindo que clientes que quitaram a 
maior parte do empréstimo são frequentemente rotulados como bons pagadores.

%TODO: tentar mostrar os valores absolutos no gráfico de barras
% barras
\begin{figure}[H]
    \centering
    \subImagem{imagens/exploratoria/finalidade.pdf}{Finalidade do empréstimo}
    \vspace{.5cm}
    \subImagem{imagens/exploratoria/categoria_renda.pdf}{Renda categorizada}
    \vspace{.5cm}
    \subImagem{imagens/exploratoria/grau.pdf}{Risco do empréstimo}
    \subImagem{imagens/exploratoria/tipo_aplicacao.pdf}{Tipo de aplicação do empréstimo}  
    \subImagemSemLegenda{imagens/exploratoria/legenda.pdf}
    
    \label{fig:exp_bar1}
    \caption{Variáveis explicativas em relação à condição do empréstimo}
\end{figure}


A Figura \ref{imagens/exploratoria/finalidade.pdf} ilustra que as categorias da variável "Finalidade" 
seguem a proporção natural da condição do empréstimo,
conforme indicado na Tabela de Condição do Empréstimo. Na Figura \ref{imagens/exploratoria/categoria_renda.pdf}, 
as categorias "Alta" e "Média"
exibem proporções menores de empréstimos ruins em comparação com a categoria "Baixa", que apresenta uma proporção de quase 10\% 
de empréstimos ruins. A Figura \ref{imagens/exploratoria/grau.pdf} revela um padrão de "cascata", indicando que à medida que o risco do 
empréstimo aumenta, a proporção de empréstimos ruins nas últimas categorias também aumenta, 
sendo a categoria G a mais afetada, com quase 25\% de empréstimos classificados como ruins. 
Na Figura \ref{imagens/exploratoria/tipo_aplicacao.pdf}, a categoria "Empréstimo conjunto" 
não registrou observações de empréstimos ruins, 
concentrando a maioria desses empréstimos na categoria "Empréstimo individual".


\begin{figure}[H]
  \centering
  \subImagem{imagens/exploratoria/pagamentos_juros.pdf}{Tipo de juros aplicado no empréstimo}
  \vspace{.5cm}
  \subImagem{imagens/exploratoria/regiao.pdf}{Região que o tomador do empréstimo vive}
  \vspace{.5cm}
  \subImagem{imagens/exploratoria/propriedade_casa.pdf}{Tipo de propriedade do tomador do empréstimo}
  \subImagem{imagens/exploratoria/prazo.pdf}{Duração do empréstimo em meses (36 ou 60 meses)}
  \subImagemSemLegenda{imagens/exploratoria/legenda.pdf}

  \caption{Variáveis explicativas em relação à condição do empréstimo}
  \label{fig:exp_bar2}
\end{figure}

Na Figura \ref{fig:exp_bar2}, o gráfico \ref{imagens/exploratoria/pagamentos_juros.pdf} 
evidencia que empréstimos obtidos sob juros compostos possuem uma proporção 
mais elevada de rotulações ruins em comparação com empréstimos sob juros simples. As Figuras \ref{imagens/exploratoria/regiao.pdf} 
e \ref{imagens/exploratoria/propriedade_casa.pdf}
destacam uma proporção natural refletida pela distribuição das categorias da variável resposta, 
conforme apresentado na Tabela \ref{tab:freq_var_res}. Já a Figura \ref{imagens/exploratoria/prazo.pdf} 
revela uma proporção mais significativa
de empréstimos ruins quando estes tendem a demorar mais para serem pagos.

\begin{table}[H]
  \centering
  \begin{tabular}{lc}
    \hline
    \textbf{Covariáveis} & \mbox{\textbf{Coeficiente de correlação}} \\ 
    \hline
    Tempo de trabalho & -0.02 \\ 
    Renda anual & -0.03 \\ 
    Valor do empréstimo & 0.00 \\ 
    Taxa de juros & 0.18 \\ 
    DTI & 0.01 \\ 
    Valor bruto pago & -0.04 \\ 
    Valor liquido pago & -0.10 \\ 
    Parcela & -0.01 \\ 
    Duração do empréstimo &  0.01 \\
    \hline
  \end{tabular}
  \caption{Valores do coeficiente de Pearson entre as covariáveis e a variável resposta}
  \label{tab:coef_corr}
  \end{table}

A partir da análise da Tabela \ref{tab:coef_corr}, nota-se que as correlações entre as 
variáveis explicativas e a variável resposta são de baixa magnitude. Os coeficientes 
calculados indicam uma relação linear fraca ou inexistente entre essas variáveis. 
Esses resultados sugerem que outros fatores ou relações não lineares podem estar desempenhando 
um papel mais significativo na explicação da variabilidade na variável resposta.
  
\begin{table}[H]
  \centering
  \begin{tabular}{lc}
    \hline
    \textbf{Covariáveis} & \mbox{\textbf{Coeficiente de contingência}} \\ 
    \hline
    Tipo de residência & 0.04 \\ 
    Tipo de aplicação & 0.01 \\ 
    Finalidade & 0.03 \\ 
    Tipo de juros & 0.14 \\ 
    Risco & 0.15 \\ 
    Região & 0.01 \\ 
    Prazo & 0.04 \\
    Renda &  0.04 \\
     \hline
  \end{tabular}
  \caption{Valores do coeficiente de contingência entre as covariáveis e a variável resposta}
  \label{tab:coef_cont}
\end{table}

Ao analisar a Tabela \ref{tab:coef_cont}, nota-se que a maioria dos coeficientes de contingência entre as 
covariáveis e a variável resposta são próximos de zero. Destaca-se que a variável "Risco"
exibe o maior valor de associação, atingindo 0.15. Entretanto, é importante ressaltar que esse valor
ainda é relativamente baixo. Os coeficientes sugerem, em geral, uma falta de associação significativa 
entre as covariáveis mencionadas e a variável resposta.


\subsection{Regressão logística}

Falar do modelo utilizado, a normalização dos dados, os resultados
métricas de avaliação e interpretação dos coeficientes


\begin{table}[H]
\centering
\begin{tabular}{lcc}
  \toprule
    \textbf{Covariáveis} & \textbf{Coeficientes} &  \textbf{Erro padrão} \\
  \midrule
        Valor líquido pago & -4.733 &  0.034 \\
        Valor bruto pago &  3.321 &  0.028 \\
        Tipo de aplicação & -1.848 &  0.106 \\
        Taxa de juros &  1.412 &  1.412 \\
        Valor do empréstimo & -1.406 &  0.039 \\
        Risco & -1.049 &  0.014 \\
        Tipo de juros & -0.462 &  -0.462  \\
        Prazo & -0.203 &  0.028 \\
        Renda anual & -0.195 &  0.010 \\
        DTI & -0.151 &  0.011 \\
        Renda categorizada & -0.111 &  0.015 \\
        Tempo de trabalho & -0.061 &  0.009 \\
        Região &  0.038 &  0.003\\
        Duração do empréstimo &  0.033 &  0.009 \\
        Tipo de residência &  0.019 &  0.003 \\
        Finalidade &  0.019 &  0.002  \\
        Parcela &  0.005 &  0.000  \\
  \bottomrule
\end{tabular}
\caption{Estimativa dos coeficientes do modelo logístico e o erro padrão associado}
\label{tab:result_model_logist}
\end{table}

Ao analisar os resultados apresentados na Tabela \ref{tab:result_model_logist}, fica evidente
que as variáveis "Valor líquido pago" e "Valor bruto pago" exercem uma influência significativa no valor final de $P(Y=1)$.
Essas duas variáveis estão diretamente associadas à quantia do empréstimo que o cliente já quitou, 
indicando sua relevância na predição do resultado. Ao calcular a Razão de chances dessas duas variáveis, temos que:


\begin{itemize}
  \item "Valor líquido pago": apresenta um RC de 0.008, o que sugere que, mantendo todas as outras variáveis constantes, a chance de o empréstimo ser classificado como bom é 125 vezes maior do que ser classificado como ruim.
  \item "Valor bruto pago": exibe um RC de 27.68, indicando que, ao manter todas as outras variáveis constantes, a chance de o empréstimo ser classificado como ruim é 27 vezes maior do que ser classificado como bom.
\end{itemize}


\imagem{imagens/resultados/report_model_logistico.png}{Matrix de confusão do modelo logístico}

Visualizando os dados da Figura \ref{fig:imagens/resultados/report_model_logistico.png} é possível analisar 
os resultados do modelo logístico no conjunto de teste. O conjunto de teste apresenta uma distribuição da variável 
resposta de com mais de 92\% dos casos como um empréstimo bom, e o restante como o empréstimo ruim.
%TODO: falar dos falsos positivos e falsos negativos


\begin{table}[H]
\centering
\begin{tabular}{lcccc}
  \hline
                &    \textbf{Precisão} & \textbf{Recall}    &   \textbf{F1-Score} &  \textbf{Tamanho da amostra} \\
  \hline
   0               &    0.928 & 0.995  &   0.961 & 163990        \\
   1               &    0.536 & 0.062 &   0.111 & 13486        \\
   Média macro     &    0.732 & 0.528  &   0.535 & 177476        \\
   Média ponderada &    0.898 & 0.925  &   0.896 & 177476        \\
   \hline
   Acurácia     &             &           &    &  0.924649\\

  \hline
\end{tabular}
\caption{Report do modelo logístico}
\label{tab:report_logist_model}
\end{table}


Com base nos dados apresentados na Tabela \ref{tab:report_logist_model}
e na Figura \ref{fig:imagens/resultados/report_model_logistico.png},
observamos que o modelo exibe uma acurácia elevada. Ele é capaz de fazer 
previsões precisas na maioria dos casos, alcançando uma taxa de 92,46\% 
de classificações corretas no conjunto de teste. No entanto, 
é crucial destacar que essa elevada acurácia é influenciada pela proporção
significativa de casos onde o empréstimo é rotulado como "bom", presente 
em mais de 92\% dos dados de teste. Como resultado, o modelo tende a
classificar uma parte considerável dos dados como "0", refletindo a
influência dessa distribuição desigual na estimação dos parâmetros do modelo logístico. 


Ao examinarmos a precisão do modelo, observamos uma taxa de acerto de 73\%
nas previsões em comparação com as rótulos reais do conjunto de teste. 
É importante ressaltar a notável precisão na categoria "Empréstimo bom", atingindo quase 92\%. 
No entanto, vale destacar que esse valor elevado está correlacionado ao desequilíbrio nos dados,
onde a classe "Empréstimo bom" é predominante.

Ao avaliar o recall do modelo logístico, observamos, em média,
valores mais baixos em comparação com a precisão. O recall médio é de 52,8\%, 
indicando que, ao analisar as porcentagens das rótulos reais, o modelo conseguiu 
acertar um pouco mais da metade delas. Esse desempenho é atribuído ao alto número 
de falsos negativos no modelo, visto que, ao considerar o total de "Empréstimos ruins" (13.486),
 o modelo acertou apenas 834 desses casos.

O F1-score acaba refletindo a real situação do modelo, pois ele balanceia os bons resultados
apresentados pela precisão com os resultados ruins do recall. O F1-score médio apresentado foi
de 52,57\%.

A avaliação global do modelo logístico revela um viés significativo, amplificado pelo desequilíbrio nos dados.
Embora o modelo tenha alcançado uma taxa geral de acerto de 92\%, sua incapacidade de distinguir
adequadamente entre "Empréstimos bons" e "Empréstimos ruins" é evidente. 
Este desempenho inferior sugere limitações na capacidade do modelo de generalizar e
discriminar efetivamente entre as categorias, indicando a necessidade de refinamentos ou 
considerações adicionais para melhorar sua robustez.


\subsection{Modelagem da rede neural}
% TODO: explicar os resultados da rede neural igual o do modelo logístico

\begin{table}[H]
  \centering
\begin{tabular}{lcccc}
  \hline
  &    \textbf{Precisão} & \textbf{Recall}    &   \textbf{F1-Score} &  \textbf{Tamanho da amostra} \\
  \hline
   0                &    0.954 & 0.904 &   0.929 & 163990        \\
   1                &    0.291 & 0.478 &   0.362 & 13486         \\
   Média macro      &    0.622 & 0.691 &   0.645 & 177476        \\
   Média ponderada  &    0.904 & 0.872 &   0.886 & 177476        \\ \hline
   Acurácia         &          &       &         & 0.872      \\
  \hline
\end{tabular}
\caption{Report da rede neural}
\label{tab:report_model_neural_ganhador}
\end{table}
\imagem{imagens/resultados/report_model_neural_ganhador.png}{Matrix de confusão da rede neural}

\subsection{Interpretação da rede neural}

- mostrar gráfico da média dos shap vs regressao logistica
\imagem{imagens/shap/media_shap.png}{Média absoluta dos valores de shap}

% - mostrar o grafico de dependencia entre valor da variável com resultado do modelo variando ela
% \imagem{imagens/shap/depend_var_res_shap.png}{Relação entre a variável X com o resultado modelo quando a mesma varia}

- mostrar 2 gráficos de shap especificos de 2 observações(pra mau pagador e pra bom pagador)
\begin{figure}[H]
  \centering
  \subImagem{imagens/shap/shap_plot_verdadeiro_positivo.png}{Valor e classificação do emprestimo}
  \vspace{.5cm}
  \subImagem{imagens/shap/shap_plot_falso_positivo.png}{WSIR residuals.}
  \vspace{.5cm}
  \subImagem{imagens/shap/shap_plot_falso_negativo.png}{WSIR predictions in the covariate space.}
 \subImagem{imagens/shap/shap_plot_verdadeiro_negativo.png}{WSIR residuals}
  
  \label{fig:enter-label}
  \caption{aloalo}
\end{figure}

- mostrar o gráfico com todos as amostras de shap(shap.plots.beeswarm)
\imagem{imagens/shap/shap_geral.png}{Valores de shap para as 80 observações utilizadas}


- mostrar o gráfico de força pra apenas uma observaçã
\imagem{imagens/shap/force_plot.png}{Gráfico de força em uma observação}

- mostrar o gráfico de força para todas as observações
\imagem{imagens/shap/force_plot_multiples_vars.png}{Gráfico de força para múltiplas observações}





\subsection{Benchmark entre regressão logística e redes neurais }


\subsubsection{Complexidade da arquitetura}

\begin{table}[H]
  \centering
  \begin{tabular}{lc}
  \hline
  \textbf{Modelo} & \textbf{Número de parâmetros} \\ \hline
  Regressão Logística & 18 \\ 
  Rede Neural & 151233 \\ \hline
  \end{tabular}
  \caption{Número de parâmetros nos modelos}
  \label{table:model_parameters}
\end{table}

% TODO: tabela com as métricas comparadas
% pode-se falar dos resultados em geral, acc, recall e f1-score etc
% e na hora de falar delas em específico falar apenas dos casos
% positivos, já que é onde o modelo está errando mais

\subsubsection{Resultado dos modelos}

\begin{table}[H]
  \centering
  \begin{tabular}{lcc}
    \hline
    \textbf{Métricas} & \textbf{Regressão logística} & \textbf{rede neural} \\
    \hline
    Falsos Positivos   & 721                     & 15703 \\
    Falsos Negativos   & 12652                   & 7039 \\
    \hline
  \end{tabular}
  \caption{Comparação dos resultados de Falsos Positivos e Falsos Negativos}
  \label{tab:false_positives_negatives}
\end{table}


\begin{table}[H]
  \centering
  \begin{tabular}{lcc}
    \hline
    \textbf{Métricas} & \textbf{Regressão logística} & \textbf{Rede neural} \\
    \hline
    Precisão (Classe 0) & 0.928081                     & \cellcolor{green!25}0.954 \\
    Precisão (Classe 1) & \cellcolor{green!25}0.536334 & 0.291 \\
    Recall (Classe 0)   & \cellcolor{green!25}0.995603 & 0.904 \\
    Recall (Classe 1)   & 0.0618419                    & \cellcolor{green!25}0.478 \\
    F1-Score (Classe 0) & \cellcolor{green!25} 0.960657  & 0.929 \\
    F1-Score (Classe 1) & 0.110897                     & \cellcolor{green!25}0.362 \\
    Acurácia            & \cellcolor{green!25}0.924649   & 0.872 \\
    \hline
  \end{tabular}
  \caption{Comparação dos resultados da Regressão logística e Rede neural}
  \label{tab:comparison_results}
\end{table}

\subsubsection{Tempo de execução}

\begin{table}[H]
\centering
\begin{tabular}{lcc}
  \hline
  \textbf{Estatísticas} & \textbf{Regressão logística} & \textbf{Rede neural} \\
  \hline
  Mínimo & 0.0 & 52.067995 \\
  Quartil 25 & 0.0 & 53.327155 \\
  Média & 0.3335619 & 59.446688 \\
  Mediana & 0.0 & 56.855202 \\
  Quartil 75 & 0.88143349 & 66.278052 \\
  Máximo & 1.50370598 & 92.03124 \\
  Variância & 0.28385463 & 59.008917 \\
  Desvio padrão & 0.5327801 & 7.681726 \\
  \hline
\end{tabular}
\caption{Tempo de predição (em ms) de cada modelo, em uma amostra com 50 observações}
\label{tab:bench_models}
\end{table}

%TODO: Tempo de interpretação
%quando montar essas informações, esclarecer que para o shap fazer as interpretações
% foi necessário primeiro definir o conjunto de treinamento 
% e depois para cada observação que vc queira interpretar
% tem o tempo gasto para se fazer isso


\newpage

% \centering
% \begin{tabular}{lcccccc}
%   & \textbf{coef} & \textbf{std err} & \textbf{z} & \textbf{P$> |$z$|$} & \textbf{[0.025} & \textbf{0.975]}  \\
% \midrule
% \textbf{duracao\_emprestimo}       &      -0.0606  &        0.005     &   -12.360  &         0.000        &       -0.070    &       -0.051     \\
% \textbf{propriedade\_casa}         &       0.0194  &        0.003     &     7.241  &         0.000        &        0.014    &        0.025     \\
% \textbf{categoria\_renda}          &      -0.1112  &        0.015     &    -7.218  &         0.000        &       -0.141    &       -0.081     \\
% \textbf{renda\_anual}              &      -0.1958  &        0.010     &   -20.148  &         0.000        &       -0.215    &       -0.177     \\
% \textbf{quantidade\_emprestimo}    &      -1.4063  &        0.039     &   -36.171  &         0.000        &       -1.483    &       -1.330     \\
% \textbf{prazo}                     &      -0.2033  &        0.028     &    -7.359  &         0.000        &       -0.257    &       -0.149     \\
% \textbf{tipo\_aplicacao}           &      -1.8479  &        0.106     &   -17.427  &         0.000        &       -2.056    &       -1.640     \\
% \textbf{finalidade}                &       0.0186  &        0.002     &     9.529  &         0.000        &        0.015    &        0.022     \\
% \textbf{pagamentos\_juros}         &      -0.4619  &        0.016     &   -29.243  &         0.000        &       -0.493    &       -0.431     \\
% \textbf{taxa\_juros}               &       1.4125  &        0.020     &    72.213  &         0.000        &        1.374    &        1.451     \\
% \textbf{grau}                      &      -1.0490  &        0.014     &   -76.099  &         0.000        &       -1.076    &       -1.022     \\
% \textbf{dti}                       &      -0.1505  &        0.011     &   -13.463  &         0.000        &       -0.172    &       -0.129     \\
% \textbf{pagamento\_bruto}          &       3.3207  &        0.028     &   118.072  &         0.000        &        3.266    &        3.376     \\
% \textbf{pagamento\_liquido}        &      -4.7329  &        0.034     &  -138.111  &         0.000        &       -4.800    &       -4.666     \\
% \textbf{parcela}                   &       0.0057  &        0.000     &    40.149  &         0.000        &        0.005    &        0.006     \\
% \textbf{regiao}                    &       0.0378  &        0.003     &    11.203  &         0.000        &        0.031    &        0.044     \\
% \textbf{duracao\_emprestimo\_dias} &       0.0326  &        0.009     &     3.704  &         0.000        &        0.015    &        0.050     \\
% \bottomrule
% \end{tabular}