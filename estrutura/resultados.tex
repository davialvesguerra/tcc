Esta seção inicia explorando a relação entre as variáveis explicativas e a variável resposta. 
Em seguida, resultados derivados tanto do modelo logístico quanto da rede neural serão detalhadamente apresentados, 
destacando a ênfase na interpretação de ambos os modelos. Além disso, será conduzido um benchmark comparativo
 entre as duas abordagens, proporcionando uma análise crítica de suas performances.

 \subsection{Análise descritiva}

\subsubsection{Condição do empréstimo}

A variável ``Condição do empréstimo" é a variável reposta desse estudo, como foi definido anteriormente.
Com isso temos o seguinte comportamento dessa variável:

\begin{table}[H]
  \centering
  \begin{tabular}{lcc}
    \hline
    \textbf{Condição do empréstimo} & \textbf{Número de observações} & \textbf{Frequência relativa} \\ \hline
    Empréstimo bom  & 819950 & 92,4\% \\
    Empréstimo ruim & 67429  & 7,59\% \\ \hline
  \end{tabular}
  \caption{Descrição da variável resposta}
  \label{tab:freq_var_res}
\end{table}

A Tabela \ref{tab:freq_var_res} mostra a distribuição da variável ``Condição do empréstimo". Uma variável composta majoritariamente 
por observações do tipo ``Empréstimo bom", presente em mais de 90\% das observações na base de dados, mostrando
que a cada 12 empréstimos rotulados como ``bons", existe 1 rotulado como ``ruim". 

\subsubsection{Relação entre as covariáveis e a variável resposta}

% boxplot
\begin{figure}[H]
  \centering
  \subImagem{imagens/exploratoria/quantidade_emprestimo.jpg}{Valor do empréstimo}
  \vspace{.5cm}
  \subImagem{imagens/exploratoria/duracao_emprestimo.jpg}{Tempo de emprego (em anos)}
  \vspace{.5cm}
  \subImagem{imagens/exploratoria/parcela.jpg}{Valor da parcela do emprésitmo}
 \subImagem{imagens/exploratoria/renda_anual.jpg}{Renda anual (em dólar)}
  
  \label{fig:exp_boxplot1}
  \caption{Boxplot das covariáveis com relação a condição do empréstimo - Parte 1}
  
\end{figure}

O comportamento da variável resposta nas Figuras \ref{fig:imagens/exploratoria/quantidade_emprestimo.jpg} 
e \ref{fig:imagens/exploratoria/parcela.jpg} demonstrou semelhanças, visto que, em ambos os casos,
não foi evidenciada uma clara diferença entre o valor do empréstimo e o valor da parcela em relação às
categorias da variável resposta. A Figura \ref{fig:imagens/exploratoria/duracao_emprestimo.jpg} 
também apresenta um comportamento semelhante entre as classes
``Empréstimo ruim" e ``Empréstimo bom", sendo que a mediana do tempo de trabalho dos clientes rotulados 
como ``Empréstimo ruim" foi inferior em comparação ao outro caso. Por fim, a Figura \ref{fig:imagens/exploratoria/renda_anual.jpg} 
indica que clientes 
com uma renda anual maior tendem a ser categorizados como ``Empréstimo bom".


\begin{figure}[H]
  \centering
  \subImagem{imagens/exploratoria/taxa_juros.jpg}{Taxa de juros do empréstimo}
  \vspace{.5cm}
  \subImagem{imagens/exploratoria/dti.jpg}{Razão entre a dívida e o salário do cliente}
  \vspace{.5cm}
  \subImagem{imagens/exploratoria/pagamento_bruto.jpg}{Valor bruto do emprétimo pago}
 \subImagem{imagens/exploratoria/pagamento_liquido.jpg}{Valor líquido do emprétimo pago}
 \caption{Boxplot das covariáveis com relação a condição do empréstimo - Parte 2}
  
 \label{fig:exp_boxplot2}
\end{figure}

A Figura \ref{fig:imagens/exploratoria/taxa_juros.jpg} evidencia uma relação entre taxas de juros elevadas e empréstimos 
considerados ruins. A Figura \ref{fig:imagens/exploratoria/dti.jpg} complementa a informação fornecida pela 
Figura \ref{fig:imagens/exploratoria/renda_anual.jpg},
indicando que clientes com renda mais elevada tem ligeira propensão a cumprir adequadamente com seus pagamentos.
As Figuras \ref{fig:imagens/exploratoria/pagamento_bruto.jpg} e  \ref{fig:imagens/exploratoria/pagamento_liquido.jpg} 
seguem padrões semelhantes, sugerindo que clientes que quitaram o empréstimo têm uma leve
tendência de serem rotulados como bons pagadores.

% barras
\begin{figure}[H]
    \centering
    \subImagem{imagens/exploratoria/finalidade.pdf}{Finalidade do empréstimo}
    \vspace{.5cm}
    \subImagem{imagens/exploratoria/categoria_renda.pdf}{Renda categorizada}
    \vspace{.5cm}
    \subImagem{imagens/exploratoria/grau.pdf}{Risco do empréstimo}
    \subImagem{imagens/exploratoria/tipo_aplicacao.pdf}{Tipo de aplicação do empréstimo}  
    \subImagemSemLegenda{imagens/exploratoria/legenda.pdf}
    
    \label{fig:exp_bar1}
    \caption{Gráfico de barras das covariáveis com relação a condição do empréstimo - Parte 1}
\end{figure}


A Figura \ref{fig:imagens/exploratoria/finalidade.pdf} ilustra que as categorias da variável ``Finalidade" 
seguem a proporção natural da condição do empréstimo,
indicado na Tabela \ref{tab:freq_var_res}. Na Figura \ref{fig:imagens/exploratoria/categoria_renda.pdf}, 
as categorias de renda ``Alta" e ``Média"
exibem proporções menores de empréstimos ruins em comparação com a categoria ``Baixa", que apresenta uma proporção de quase 10\% 
de empréstimos ruins. A Figura \ref{fig:imagens/exploratoria/grau.pdf} revela um padrão de crescente na proporção
dos empréstimos ruins, indicando que à medida que o risco do 
empréstimo aumenta, a proporção de empréstimos ruins nas últimas categorias também aumenta, 
sendo a categoria G a mais afetada, com quase 25\% de empréstimos classificados como ruins. 
Na Figura \ref{fig:imagens/exploratoria/tipo_aplicacao.pdf}, a categoria ``Empréstimo conjunto" 
não registrou observações de empréstimos ruins, 
concentrando a maioria desses empréstimos na categoria ``Empréstimo individual".


\begin{figure}[H]
  \centering
  \subImagem{imagens/exploratoria/pagamentos_juros.pdf}{Tipo de juros aplicado no empréstimo}
  \vspace{.5cm}
  \subImagem{imagens/exploratoria/regiao.pdf}{Região que o tomador do empréstimo vive}
  \vspace{.5cm}
  \subImagem{imagens/exploratoria/propriedade_casa.pdf}{Tipo de propriedade do tomador do empréstimo}
  \subImagem{imagens/exploratoria/prazo.pdf}{Duração do empréstimo em meses (36 ou 60 meses)}
  \subImagemSemLegenda{imagens/exploratoria/legenda.pdf}

  \caption{Gráfico de barras das covariáveis com relação a condição do empréstimo - Parte 2}
  \label{fig:exp_bar2}
\end{figure}

Na Figura \ref{fig:exp_bar2}, o gráfico \ref{fig:imagens/exploratoria/pagamentos_juros.pdf} 
evidencia que empréstimos obtidos sob juros compostos possuem uma proporção 
mais elevada de rotulações ruins em comparação com empréstimos sob juros simples. As Figuras \ref{fig:imagens/exploratoria/regiao.pdf} 
e \ref{fig:imagens/exploratoria/propriedade_casa.pdf}
apresentam uma proporção natural refletida pela distribuição das categorias da variável resposta, 
conforme apresentado na Tabela \ref{tab:freq_var_res}. Já a Figura \ref{fig:imagens/exploratoria/prazo.pdf} 
revela uma proporção ligeiramente maior
de empréstimos ruins quando estes tem uma duração maior.


\subsection{Regressão logística}

Como visto na Equação \ref{eq:model_logistic}, a fórmula do modelo logístico é dada por:

\begin{equation}
  P(Y=1| x_1, x_2, ..., x_k) = \frac{1}{1 + e^{-(\beta_0 + x_{1}\beta_1 + x_{2}\beta_2 + \ldots + x_{k}\beta_k)}}
\end{equation}

Sendo Y=1 a classificação de clientes com ``Empréstimos ruins" e Y=0, os clientes com ``Empréstimos bons". 
Dessa forma, os seguintes resultados foram encontrados:


\begin{table}[H]
\centering
\begin{tabular}{lcc}
  \toprule
    \textbf{Covariáveis} & \textbf{Coeficientes} &  \textbf{Erro padrão} \\
  \midrule
        Valor líquido pago & -4.733 &  0.034 \\
        Valor bruto pago &  3.321 &  0.028 \\
        Tipo de aplicação & -1.848 &  0.106 \\
        Taxa de juros &  1.412 &  1.412 \\
        Valor do empréstimo & -1.406 &  0.039 \\
        Risco & -1.049 &  0.014 \\
        Tipo de juros & -0.462 &  -0.462  \\
        Prazo & -0.203 &  0.028 \\
        Renda anual & -0.195 &  0.010 \\
        DTI & -0.151 &  0.011 \\
        Renda categorizada & -0.111 &  0.015 \\
        Tempo de trabalho & -0.061 &  0.009 \\
        Região &  0.038 &  0.003\\
        Duração do empréstimo &  0.033 &  0.009 \\
        Tipo de residência &  0.019 &  0.003 \\
        Finalidade &  0.019 &  0.002  \\
        Parcela &  0.005 &  0.000  \\
  \bottomrule
\end{tabular}
\caption{Estimativa dos coeficientes do modelo logístico e o erro padrão associado (todas as variáveis foram significativas ao nível de significância de 5\%)}
\label{tab:result_model_logist}
\end{table}

Considerando que a escala das covariáveis foi padronizada para ter média 0 e desvio padrão unitário, é possível perceber que,
com os resultados apresentados na Tabela \ref{tab:result_model_logist}, fica evidente
que as variáveis ``Valor líquido pago" e ``Valor bruto pago" exercem uma influência na maior probabilidade de $P(Y=1)$.
Essas duas variáveis estão diretamente associadas à quantia do empréstimo que o cliente já quitou, 
indicando sua relevância na predição do resultado. Ao calcular a Razão de chances dessas duas variáveis, temos que:


\begin{itemize}
  \item ``Valor líquido pago": $\exp(-4.733) = 0.008$, o que sugere que, mantendo todas as outras variáveis constantes, aumentar em 1 desvio padrão o valor líquido pago, torna 125 maior a chance do empréstimo ser classificado como ruim.
  \item ``Valor bruto pago": $\exp(3.321) = 0.008$, indicando que, ao manter todas as outras variáveis constantes, aumentar em 1 desvio padrão o valor bruto pago, torna 27 maior a chance do empréstimo ser classificado como bom.
\end{itemize}


\imagem{imagens/resultados/report_model_logistico.png}{Matrix de confusão do modelo logístico}

 O conjunto de teste apresenta uma distribuição da variável 
resposta de com mais de 92\% dos casos como um empréstimo bom, e o restante como o empréstimo ruim.


O modelo logístico fez um pouco mais de 30\% de previsões da classe ``1". 
Esse cenário resultou em um alto número de Falsos Positivos. 
Em contrapartida, o número de Falsos Negativos ficou menor, resultando em menos de
3\% das previsões da amostra de teste sendo incorretamente rotuladas como classe ``0".



\begin{table}[H]
\centering
\begin{tabular}{lcccc}
  \hline
                &    \textbf{Precisão} & \textbf{Recall}    &   \textbf{F1-Score} &  \textbf{Tamanho da amostra} \\
  \hline
   0               &    0.928 & 0.716    &   0.823 & 163990        \\
   1               &    0.170 & 0.708 &   0.274 & 13486        \\
   Média macro     &    0.569 & 0.712  &   0.548 & 177476        \\
   Média ponderada &    0.907 & 0.715  &   0.781 & 177476        \\
   \hline
   Acurácia     &             &           &    &  0.715\\

  \hline
\end{tabular}
\caption{Métricas de avaliação do modelo logístico}
\label{tab:report_logist_model}
\end{table}


Com base nos dados apresentados na Tabela \ref{tab:report_logist_model}
e na Figura \ref{fig:imagens/resultados/report_model_logistico.png},
observamos que o modelo exibe uma acurácia elevada. Ele é capaz de fazer 
previsões precisas em um número considerável dos casos, alcançando uma taxa de 92,46\% 
de classificações corretas no conjunto de teste. 

Ao examinarmos a precisão do modelo, observamos uma taxa de acerto de 56,9\%
nas previsões em comparação com as rótulos reais do conjunto de teste. 
É importante ressaltar a notável precisão na categoria ``Empréstimo bom", atingindo quase 92\%. 
No entanto, vale destacar que esse valor elevado está correlacionado ao desequilíbrio nos dados,
onde a classe ``Empréstimo bom" é predominante, pois ao observarmos as precisão da classe 1, é possível notar um
valor menor que 20\%.

Ao avaliar o Recall do modelo logístico, observamos, em média,
valores maiores em comparação com a precisão. O recall médio é de 71,2\%, 
indicando que, ao analisar as porcentagens das rótulos reais, o modelo conseguiu 
acertar um valor considerável. Esse desempenho é atribuído ao baixo número 
de falsos negativos no modelo, visto que, ao considerar o total de ``Empréstimos ruins" (13.486),
 o modelo acertou apenas 9554 desses casos.

O F1-score acaba refletindo a real situação do modelo, pois ele balanceia os resultados
de ambas as métricas. O F1-score médio apresentado foi
de 54,8\%.

O modelo apresentou resultados interessantes, dado a sua arquitetura não tão robusta, uma taxa geral de acerto de 71,5\%.
Na classe minoritária, a classe ``1", o modelo foi pouco preciso, acertando menos de 20\% dos casos.
Este desempenho inferior sugere limitações na capacidade do modelo, indicando a necessidade de refinamentos ou 
considerações adicionais para melhorar seu melhor desempenho preditivo.


\subsection{Rede neural}

Esse seção vai abordar os resultados obtidos pelo modelo de rede neural.


\imagem{imagens/resultados/report_model_neural_ganhador.png}{Matrix de confusão da rede neural}


A Figura \ref{fig:imagens/resultados/report_model_neural_ganhador.png} indica que aproximadamente 
12\% das previsões do modelo na amostra de teste consistem em empréstimos ruins, uma estimativa ligeiramente distante 
da distribuição real apresentada na Tabela \ref{tab:freq_var_res}.
Por outro lado, o número de empréstimos
classificados como ``Empréstimos bons", diminuiu, representando mais de 84\% da amostra de teste. 


\begin{table}[H]
  \centering
\begin{tabular}{lcccc}
  \hline
  &    \textbf{Precisão} & \textbf{Recall}    &   \textbf{F1-Score} &  \textbf{Tamanho da amostra} \\
  \hline
   0                &    0.954 & 0.904 &   0.929 & 163990        \\
   1                &    0.291 & 0.478 &   0.362 & 13486         \\
   Média macro      &    0.622 & 0.691 &   0.645 & 177476        \\
   Média ponderada  &    0.904 & 0.872 &   0.886 & 177476        \\ \hline
   Acurácia         &          &       &         & 0.872      \\
  \hline
\end{tabular}
\caption{Métricas de avaliação da rede neural}
\label{tab:report_model_neural_ganhador}
\end{table}


Os resultados apresentados na Tabela \ref{tab:report_model_neural_ganhador} indicam um desempenho satisfatório
do modelo, especialmente em termos de acurácia e métricas avaliadas para a classe ``0". 
Contudo, ao comparar esses resultados com as métricas da classe ``1", percebe-se que o modelo ainda
é influenciado pelo elevado número de observações na classe ``0". Por outro lado, uma análise mais detalhada 
do Recall da classe ``1" revela que o modelo conseguiu reduzir significativamente o número de Falsos Negativos,
identificando corretamente quase metade dos empréstimos considerados ruins na base de dados. Por outro lado,
a precisão da classe ``1" diminuiu, refletindo que menos de 30\% 
das previsões do modelo foram corretas nesse contexto, como evidenciado na Tabela \ref{tab:report_model_neural_ganhador}.


Em termos gerais, as decisões relacionadas à arquitetura do modelo, seus hiperparâmetros, 
estratégia de treinamento e outros fatores contribuíram para que o modelo de redes neurais realizasse previsões
de boa qualidade, lidando melhor com o desbalanceamento dos dados.  Portanto, buscar aprimorar ainda mais essa 
arquitetura pode ser uma abordagem promissora na busca por resultados ainda melhores.

\subsection{Interpretação da rede neural}


Após definir o modelo de rede neural e examinar seus resultados, esta seção aborda a interpretação do modelo. 
Para isso, foram construídos gráficos com base nos resultados do SHAP, destacando as variáveis de maior importância
 no resultado final. Utilizando uma amostra de 80 observações, o valor de SHAP foi calculado para cada observação, 
 permitindo uma análise tanto individual quanto conjunta dessa amostra.

\imagem{imagens/shap/media_shap.png}{Média absoluta dos valores de SHAP de cada covariável}


O Gráfico \ref{fig:imagens/shap/media_shap.png} exibe a média absoluta dos valores de SHAP para cada covariável nas
80 observações consideradas. Essa representação oferece uma visão sobre quais variáveis o modelo considerou mais
relevantes durante as predições, destacando a magnitude da contribuição de cada variável. Vale notar que, por 
se tratar de valores absolutos, o gráfico não proporciona informações sobre se a contribuição de cada variável
é positiva ou negativa. Entretanto, os próximos gráficos irão elucidar essa questão ao detalhar a contribuição 
específica de cada variável.

Ao analisar cada variável no gráfico, destaca-se que ``Total líquido pago" e ``Total bruto pago" são as que mais
contribuíram para o resultado final do modelo. O gráfico enfatiza a relevância de nove variáveis, omitindo 
o restante devido à sua baixa contribuição. Ao observar as variáveis omitidas, percebe-se que, somadas, 
suas contribuições aproximam-se de 0.01, evidenciando a sua baixa influência no resultado final do modelo de rede neural.

Os próximos gráficos proporcionam a visualização dos valores SHAP para cada variável em observações individuais, ilustrando
cenários distintos que podem aparecer nos problemas de classificação binária. Os gráficos também revelam os valores específicos
observados para cada covariável de uma dada observação.

As Figuras \ref{fig:imagens/shap/shap_plot_verdadeiro_positivo.png} e \ref{fig:imagens/shap/shap_plot_verdadeiro_negativo.png} 
mostram casos onde o modelo acertou suas predições, tanto para casos de ``Empréstimos ruins" (Figura \ref{fig:imagens/shap/shap_plot_verdadeiro_positivo.png})
quanto para ``Empréstimos bons" (Figura \ref{fig:imagens/shap/shap_plot_verdadeiro_negativo.png}).

\imagem{imagens/shap/shap_plot_verdadeiro_positivo.png}{Valor de SHAP para uma observação do tipo Verdadeiro Positivo}

Na Figura \ref{fig:imagens/shap/shap_plot_verdadeiro_positivo.png}, temos que, no eixo y, 
os valores observados para as covariáveis dessa observação específica do banco de dados, enquanto que nas barras
coloridas do gráfico, temos indicado o valor de SHAP para cada covariável (contribuições positivas, em vermelho,
 e contribuições negativas, em azul). A soma de todos os valores SHAP retorna a predição dessa observação, nesse exemplo,
 $f(x) = 0,909$, conforme indicado no eixo x.

No cenário acima, em que o modelo acertou a classificação de 
um empréstimo como ruim, observa-se que as variáveis que mais contribuíram foram as mesmas observadas na Figura \ref{fig:imagens/shap/media_shap.png}, 
comportamento esse que vai prevalecer nos demais casos.
Ao analisar os valores observados dessas duas covariáveis, nota-se que, para esse cliente específico, ainda resta um montante significativo do empréstimo a ser pago, 
totalizando quase 90\% do valor líquido pendente e quase 50\% do valor bruto pendente. Uma diferença muito grande
entre o valor do empréstimo e o que falta a ser pago pode ser um dos fatores que está ocasionando a rotulação dessa observação
como ``Empréstimos ruins". 

\imagem{imagens/shap/shap_plot_verdadeiro_negativo.png}{Valor de SHAP para uma observação do tipo Verdadeiro Negativo}

Ao analisar a Figura \ref{fig:imagens/shap/shap_plot_verdadeiro_negativo.png}, que representa o cenário em que o modelo
acertou a rotulação de um empréstimo bom, nota-se um valor final bem próximo de  0 (f(x) = 0.003). Indícios 
de que o modelo teve mais confiança ao realizar essa predição. 
Ao analisar os valores de cada variável, é possível perceber que o cliente já está finalizando ou finalizou o empréstimo,
dado que as variáveis de pagamento do empréstimo chegaram no valor real do empréstimo.


Analisando as Figuras \ref{fig:imagens/shap/shap_plot_verdadeiro_positivo.png} e \ref{fig:imagens/shap/shap_plot_verdadeiro_negativo.png},
é possível observar a 
flexibilidade do modelo em lidar com diferentes valores da mesma covariável. Quando o modelo encontra valores que indicam um 
empréstimo ruim, atribui valores positivos para a contribuição das variáveis ``Total líquido pago" e ``Total bruto pago", aproximando-as de 1. Da mesma forma, para 
empréstimos bons, o modelo adiciona uma contribuição negativa, direcionando o resultado para 0. Isso demonstra como o modelo 
responde de forma dinâmica e não linear às variações nos valores das variáveis.


\imagem{imagens/shap/shap_plot_falso_positivo.png}{Valor de SHAP para uma observação do tipo Falso Positivo}


Ao analisar um exemplo em que o modelo erroneamente classifica como ``Empréstimo ruim", conforme ilustrado na Figura
 \ref{fig:imagens/shap/shap_plot_falso_positivo.png}, destacam-se as características já discutidas nas interpretações
  anteriores. O resultado do modelo foi muito próximo do limiar de decisão, que é 0,5, indicando que o modelo teve 
  uma maior indecisão ao realizar a predição desse caso.


\imagem{imagens/shap/shap_plot_falso_negativo.png}{Valor de SHAP para uma observação do tipo Falso Negativo}

Observando a Figura \ref{fig:imagens/shap/shap_plot_falso_negativo.png}, que retrata um cenário em que o modelo
 classifica de forma equivocada como ``Empréstimo bom", é possível perceber um comportamento de divergência entre 
 as  duas covariáveis que mais contribuem com o resultado do modelo. Esse padrão de divergência não foi observado nos
  casos anteriores, e pode indicar uma incerteza na predição.


\imagem{imagens/shap/shap_geral.png}{Valores de SHAP para as 80 observações utilizadas}


A Figura \ref{fig:imagens/shap/shap_geral.png} apresenta o comportamento do valor de SHAP para cada covariável
entre as 80 observações utilizadas no experimento. No gráfico, o eixo x refere-se à distribuição do valor de SHAP,
enquanto o eixo y representa cada variável. O eixo das cores ilustra a distribuição dos valores observados da covariável, 
sendo que cores mais quentes indicam valores maiores e cores mais frias indicam valores menores.

Ao analisar as variáveis que mais influenciam o resultado do modelo, observa-se uma distribuição mais dispersa para o valor de SHAP.
 Para a primeira variável, elevados valores da covariável ``Total líquido pago" tendem a impactar negativamente no resultado do modelo,
 ou seja, diminuir a probabilidade de classificação como um ``Empréstimo Ruim"
 enquanto a segunda variável apresenta o comportamento oposto, com valores mais baixos do ``Total bruto pago",
  contribuindo negativamente no
  resultado do modelo.


\imagem{imagens/shap/force_plot.png}{Gráfico de força do SHAP para uma observação}


A Figura \ref{fig:imagens/shap/force_plot.png} ilustra a ``força" de contribuição de cada variável no modelo.
 Este gráfico proporciona uma visão clara das variáveis que tiveram impacto positivo e negativo no resultado dessa observação 
 específica.
  Cada barra representa a intensidade da contribuição de uma variável específica, sendo que aquelas com maior 
  influência concentram-se no centro, enquanto as de menor influência ficam nas extremidades. 
  A figura é centrada no valor final predito pelo modelo, que, neste caso, é observado como $f(x) = 0.3$, 
  indicando um empréstimo classificado como bom.

\imagem{imagens/shap/force_plot_multiples_vars.png}{Gráfico de força para múltiplas observações}


A Figura \ref{fig:imagens/shap/force_plot_multiples_vars.png} é uma generalização da Figura \ref{fig:imagens/shap/force_plot.png},
considerando 160 observações. Este gráfico é um recorte obtido de uma visualização dinâmica gerada pelo pacote SHAP.
 Devido à natureza dinâmica do gráfico original, ao transformá-lo em uma imagem estática, parte de sua capacidade de 
 representação foi limitada. No entanto, é possível observar alguns padrões nesse recorte. Por exemplo, 
 a predominância do vermelho em determinado recorte vertical, indica a previsão de um ``Empréstimo ruim".
  Da mesma forma, que a predominância da cor azul indica predições próximas de 0.


\subsection{Benchmark entre regressão logística e redes neurais }

Ao obter interpretações do modelo de redes neurais, abre-se a possibilidade de realizar comparações 
entre esses modelos com os regressão logística no âmbito interpretativo. Até recentemente, 
essa capacidade de interpretação estava exclusivamente associada ao modelo logístico em comparação com o modelo de redes neurais.
Essa análise comparativa se soma às comparações já realizadas entre os modelos, englobando aspectos como arquitetura,
tempo de treinamento,
tempo de predição e os resultados gerados por ambos os modelos. Os resultados abaixo evidenciam essas comparações.

\subsubsection{Complexidade da arquitetura}

\begin{table}[H]
  \centering
  \begin{tabular}{lc}
  \hline
  \textbf{Modelo} & \textbf{Número de parâmetros} \\ \hline
  Regressão Logística & 18 \\ 
  Rede Neural & 151233 \\ \hline
  \end{tabular}
  \caption{Número de parâmetros}
  \label{table:model_parameters}
\end{table}

Ao examinar a Tabela \ref{table:model_parameters}, destaca-se a significativa disparidade na
complexidade entre os modelos. Enquanto o modelo logístico possui apenas um parâmetro para cada covariável,
a rede neural apresenta mais de 8400 parâmetros para cada parâmetro do modelo logístico,  
sendo esses distribuídos nos neurônios e camadas da rede.

\subsubsection{Resultado dos modelos}

\begin{table}[H]
  \centering
  \begin{tabular}{lcc}
    \hline
    \textbf{Métricas} & \textbf{Regressão logística} & \textbf{Rede neural} \\
    \hline
    Falsos Positivos   & 46560                     & 15703 \\
    Falsos Negativos   & 3930                   & 7039 \\
    \hline
  \end{tabular}
  \caption{Comparação dos resultados de Falsos Positivos e Falsos Negativos}
  \label{tab:false_positives_negatives}
\end{table}

Ao analisar os casos em que os modelos cometeram erros, conforme apresentado na Tabela \ref{tab:false_positives_negatives}, 
é evidente que, em geral, o modelo logístico cometeu mais erros do que a rede neural. O modelo logístico registrou um total 
de mais de 50 mil classificações incorretas, enquanto a rede neural teve pouco mais de 18 mil erros.

Os resultados a seguir vão identificar melhor como foi o processo preditivo dos dois modelos no conjunto de teste.

\begin{table}[H]
  \centering
  \begin{tabular}{lcc}
    \hline
    \textbf{Métricas} & \textbf{Regressão logística} & \textbf{Rede neural} \\
    \hline
    Precisão (Classe 0) & 0.928                     & \cellcolor{green!25}0.954 \\
    Precisão (Classe 1) & 0.170                     &\cellcolor{green!25}0.291 \\
    Recall   (Classe 0) & 0.716                     & \cellcolor{green!25}0.904 \\
    Recall   (Classe 1) & \cellcolor{green!25}0.708 & 0.478 \\
    F1-Score (Classe 0) &  0.823                    & \cellcolor{green!25}0.929 \\
    F1-Score (Classe 1) & 0.274                     & \cellcolor{green!25}0.362 \\
    Acurácia            & 0.715                     & \cellcolor{green!25}0.872 \\
    \hline
  \end{tabular}
  \caption{Comparação dos resultados da Regressão logística e Rede neural. Em verde, o modelo que obteve o melhor resultado na respectiva métrica.}
  \label{tab:comparison_results}
\end{table}



Ao examinar a Tabela \ref{tab:comparison_results}, é possível observar que a rede neural apresentou melhores
resultados em quase todas as métricas. Esse desempenho destacado
da regressão logística pode ter ocorrido devido a sua arquitetura. Este modelo, ao ser treinado
com dados desbalanceados, enfrenta limitações devido à sua arquitetura menos complexa.

Contudo, ambos os modelo apresentaram um resultado ruim nos Falsos Positivos, métrica importante no contexto 
financeiro, pois permite identificar os empréstimos ruins classificados de maneira errônea.



\begin{table}[H]
\centering
\begin{tabular}{lcc}
  \hline
  \textbf{Covariáveis}               &   \parbox{4cm}{\centering\textbf{Coeficientes da \\ regressão logística}} &  
  \parbox{5cm}{ \centering\textbf{Médias absoluta \\ dos valores de SHAP}} \\
  \hline
   Total líquido pago    &         -4.733 & 0.23363 \\
   Total bruto pago      &          3.321 & 0.22619 \\
   Valor do empréstimo   &         -1.406 & 0.06298 \\
   Renda anual           &         -0.195 & 0.06132 \\
   Tempo de trabalho &         -0.061 & 0.03105 \\
   Parcela               &         0.005 & 0.02586 \\
   DTI                   &         -0.151 & 0.00242 \\
   Taxa de juros         &          1.412 & 0.00232 \\
   Risco                  &         -1.049 & 0.00169 \\
   Duração do empréstimo &          0.033 & 0.00166 \\
   Tipo de moradia       &         0.019 & 0.00117 \\
   Finalidade            &         0.019 & 0.00094 \\
   Tipo de juros        &         -0.462 & 0.00077 \\
   Região                &          0.038 & 0.00071 \\
   Prazo                 &         -0.203 & 0.00068 \\
   Renda categorizada    &         -0.111 & 0.00058 \\
   Tipo da aplicação     &         1.848 & 0       \\
  \hline
  \end{tabular}
  \caption{Relação entre os coeficientes da regressão logística com os valores absolutos de SHAP}
  \label{tab:coef_vs_shap}
\end{table}

A Tabela \ref{tab:coef_vs_shap} realiza uma comparação entre os coeficientes do modelo logístico
 e a média dos valores absolutos de SHAP, sendo esse último calculado com base em uma amostra composta por 169 observações. 
 A média dos valores de SHAP proporciona uma medida da magnitude das contribuições das variáveis,
  apresentando uma ordem de importância que pode ser comparada com a dos coeficientes da regressão logística.


 Ambos os modelos exibiram semelhança nas duas variáveis que mais impactam o resultado final.
  No entanto, ao classificar as demais variáveis em ordem de importância,
  observa-se uma divergência significativa no grau de influência que esses valores exercem.

\subsubsection{Tempo de execução}


Examinar o tempo de predição é importante na avaliação da utilidade prática do modelo, quase tão significativo
 quanto as métricas de desempenho do modelo.

\begin{table}[H]
\centering
\begin{tabular}{lcc}
  \hline
  \textbf{Estatísticas} & \textbf{Regressão logística} & \textbf{Rede neural} \\
  \hline
  Mínimo & 0.0 & 52.06 \\
  Quartil 25 & 0.0 & 53.32 \\
  Média & 0.33 & 59.44 \\
  Mediana & 0.0 & 56.85 \\
  Quartil 75 & 0.88 & 66.27 \\
  Máximo & 1.50 & 92.03 \\
  Desvio padrão & 0.53 & 7.68 \\
  \hline
\end{tabular}
\caption{Tempo de predição (em ms) de cada modelo, em uma amostra com 50 observações. }
\label{tab:bench_models}
\end{table}

A Tabela \ref{tab:bench_models} destaca a extrema eficiência do modelo logístico
 em comparação com o modelo de redes neurais. O modelo logístico demonstrou um tempo 
 de predição mais curto, com a predição mais demorada levando apenas 1,5 milissegundos,
  sendo mais de 60 vezes mais rápido do que a predição mais demorada do modelo de redes neurais.

\begin{table}[H]
\centering
\begin{tabular}{cc}
\hline
\textbf{Nº de observações} & \textbf{Tempo de execução} \\
\hline
1  & 317s \\
80 & 7.07hrs \\
\hline
\end{tabular}
\caption{Tempo de execução para realizar a interpretação das variáveis}
\label{tab:tempo_execucao_shap}
\end{table}
\newpage


Ao contrário do modelo logístico, onde não há um tempo de execução associado à interpretação, 
pois a interpretação é derivada dos coeficientes estimados, a situação é diferente no modelo de redes neurais. 
A interpretação de uma única observação demandou mais de 5 minutos, e o cálculo das interpretações para as 80
observações utilizadas nos resultados anteriores exigiu mais de 7 horas. Essa diferença substancial de tempo
destaca a complexidade computacional envolvida na interpretação de modelos mais elaborados, como redes neurais.

% \centering
% \begin{tabular}{lcccccc}
%   & \textbf{coef} & \textbf{std err} & \textbf{z} & \textbf{P$> |$z$|$} & \textbf{[0.025} & \textbf{0.975]}  \\
% \midrule
% \textbf{duracao\_emprestimo}       &      -0.0606  &        0.005     &   -12.360  &         0.000        &       -0.070    &       -0.051     \\
% \textbf{propriedade\_casa}         &       0.0194  &        0.003     &     7.241  &         0.000        &        0.014    &        0.025     \\
% \textbf{categoria\_renda}          &      -0.1112  &        0.015     &    -7.218  &         0.000        &       -0.141    &       -0.081     \\
% \textbf{renda\_anual}              &      -0.1958  &        0.010     &   -20.148  &         0.000        &       -0.215    &       -0.177     \\
% \textbf{quantidade\_emprestimo}    &      -1.4063  &        0.039     &   -36.171  &         0.000        &       -1.483    &       -1.330     \\
% \textbf{prazo}                     &      -0.2033  &        0.028     &    -7.359  &         0.000        &       -0.257    &       -0.149     \\
% \textbf{tipo\_aplicacao}           &      -1.8479  &        0.106     &   -17.427  &         0.000        &       -2.056    &       -1.640     \\
% \textbf{finalidade}                &       0.0186  &        0.002     &     9.529  &         0.000        &        0.015    &        0.022     \\
% \textbf{pagamentos\_juros}         &      -0.4619  &        0.016     &   -29.243  &         0.000        &       -0.493    &       -0.431     \\
% \textbf{taxa\_juros}               &       1.4125  &        0.020     &    72.213  &         0.000        &        1.374    &        1.451     \\
% \textbf{grau}                      &      -1.0490  &        0.014     &   -76.099  &         0.000        &       -1.076    &       -1.022     \\
% \textbf{dti}                       &      -0.1505  &        0.011     &   -13.463  &         0.000        &       -0.172    &       -0.129     \\
% \textbf{pagamento\_bruto}          &       3.3207  &        0.028     &   118.072  &         0.000        &        3.266    &        3.376     \\
% \textbf{pagamento\_liquido}        &      -4.7329  &        0.034     &  -138.111  &         0.000        &       -4.800    &       -4.666     \\
% \textbf{parcela}                   &       0.0057  &        0.000     &    40.149  &         0.000        &        0.005    &        0.006     \\
% \textbf{regiao}                    &       0.0378  &        0.003     &    11.203  &         0.000        &        0.031    &        0.044     \\
% \textbf{duracao\_emprestimo\_dias} &       0.0326  &        0.009     &     3.704  &         0.000        &        0.015    &        0.050     \\
% \bottomrule
% \end{tabular}